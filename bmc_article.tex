\documentclass{bmcart}
\usepackage[utf8]{inputenc} %unicode support



\def\includegraphic{}
\def\includegraphics{}



%%% Put your definitions there:
\startlocaldefs
\endlocaldefs


%%% Begin ...
\begin{document}

%%% Start of article front matter
\begin{frontmatter}

\begin{fmbox}
\dochead{Proyecto XXXX}



\title{TITULO}


\end{fmbox}


\end{frontmatter}



El modelo cosmológico estándar $\Lambda CDM$ ha logrado describir una gran 
variedad de hechos observacionales del universo a gran escala tales como, la 
radiación cósmica de fondo, homogeneidad e isotropía de la distribución de 
galaxias a gran escala, sin embargo, a pequeña escala ($\leq 100\, Mpc$) la 
distribución de galaxias no es homogénea, se observan acumulaciones de 
galaxias, vacíos y densos filamentos. En este modelo las galaxias se forman 
como resultado de partículas de materia oscura que se mueven muy lentamente en 
el universo temprano dando paso al crecimiento de perturbaciones amplificadas 
por la gravedad en forma de halos de materia oscura, estas galaxias se forman 
luego del enfriamiento y condensación de materia barionica al interior de estos 
halos de materia oscura que posteriormente, luego de sucesivos procesos de 
mezclas (debido a interacciones gravitacionales) se forman cúmulos y 
supercúmulos de galaxias (crecimiento jerárquico de estructura). \\

La distribución de galaxias no es aleatoria. Ellas se forman en regiones 
altamente densas y relativamente frías, la distribución espacial de las 
galaxias se ha mostrado que no es trazadora de toda la materia del universo, es 
decir, las galaxias no son trazadoras de la distribución de materia subyacente 
del universo a gran escala \cite{1984ApJ...284L...9K}; la distribución de 
materia en el universo (materia oscura) se encuentra sesgada respecto a la 
distribución de galaxias, este sesgo o bias esta relacionado con complejos 
procesos físicos que regulan la formación de estructura a gran escala y está 
también relacionado con la restricción de parámetros cosmológicos en el 
$\Lambda CDM$. El estudio de la relación entre las galaxias y los halos de 
materia oscura es fundamental para el entendimiento de la formación de 
galaxias.\\

La distribución de galaxias contiene mucha información, una de las formas más 
usadas para extraer dicha información es caracterizar la distribución espacial 
de galaxias de los grandes muestreos mediante la función de 
correlación\footnote{O su equivalente, el espectro de potencias} de dos puntos 
$\xi(r)$. Esta función de correlación mide el exceso de probabilidad sobre una 
distribución aleatoria de encontrar una galaxia a una distancia $r$ de 
cualquier otra galaxia arbitaria 
\cite{1980lssu.book.....P}\cite{1993A&A...280....5M}. Esta función se puede 
representar mediante una ley de potencias 
\cite{1983ApJ...267..465D}\cite{1993A&A...280....5M}: 

$$\xi(r)=\left(\frac{r}{r_0}\right)^{-\gamma}$$

donde $\gamma$ es la pendiente de la relación y $r_0$  es la longitud de 
correlación, físicamente $r_0$ indica que únicamente el par correlacionado a 
pequeña escala tiene suficiente potencia para ser interpretado como una huella 
de perturbaciones NO lineales \cite{1993A&A...280....5M}. Los cúmulos de 
galaxias tienen longitudes de correlación más grandes que las galaxias ya que 
lo cúmulos se encuentran en halos mas masivos, sesgados y fuertemente 
correlacionados, es decir, la función de correlación de dos puntos depende 
también de la masa de los objetos. Una de las explicaciones  a la variabilidad 
de $r_0$ es debido a que los catálogos de galaxias contienen grandes 
in-homogeneidades, grandes estructuras, filamentos y vacíos 
\cite{2005MNRAS.357..608Y}, algunos autores interpretan esta variación como 
prueba observacional directa de un universo fractal \cite{1987PhyA..144..257P}. 
En muchos casos se usa el corrimiento al rojo ($z$) como indicador de distancia 
y se le asocia una función de correlación, llamada función de correlación en 
el espacio del corrimiento al rojo. \\

En el $\Lambda CDM$ las galaxias se forman y evolucionan en el interior de 
halos de materia oscura y adicionalmente, la distribución galaxias (trazadores) 
se encuentra sesgada respecto a la distribución de materia oscura en el 
universo \cite{1984ApJ...284L...9K}, es decir, la distribución espacial de las 
galaxias difiere de la distribución de materia oscura, sin embargo, se puede 
inferir la distribución de materia oscura desde la caracterización del 
acumulamiento de las galaxias teniendo en cuenta ese parámetro de sesgo.  Un 
método común para interpretar las medidas del acumulamiento de galaxias 
observado es la Distribución de Ocupación de Halos (HOD) 
\cite{2000MNRAS.318..203S}\cite{2007ApJ...667..760Z}\cite{2016arXiv161001991V}. 
El HOD caracteriza la relación entre las galaxias y los halos de materia oscura 
en términos de la distribución de probabilidad $P(N|M_h)$ que un halo de masa 
virial $M_h$ contenga $N$ galaxias de un tipo dado, la suposición estándar de 
esta distribución es que únicamente la masa del halo es suficiente para 
determinar la población de galaxias de ese halo (y por lo tanto su acumulación) 
\cite{2016arXiv161001991V}\cite{2018ApJ...853...84Z}, sin embargo, en años 
recientes se han encontrado dependencias adicionales, la distribución espacial 
de los halos no depende únicamente de su masa, también depende de los detalles 
de su historia de ensamblaje, tiempo de formación del halo, concentración, 
etc..., esto se conoce como el Halo Assembly Bias 
\cite{2017ApJ...848...60Y}\cite{doi:10.1111/j.1745-3933.2007.00292.x}\cite{2005MNRAS.363L..66G}\cite{2016PhRvL.116d1301M}.\\


El origen del assembly bias aun no es claro, 
\cite{doi:10.1111/j.1365-2966.2009.15271.x} sugiere que los halos de baja masa 
en los filamentos son impulsados por la supresión de las mareas en razón del 
crecimiento del halo en la vecindad de un halo masivo vecino. Si las 
propiedades galácticas están casi correlacionadas con las historias de 
formación del los halos, el assembly bias puede verse reflejado en la 
distribución de galaxias \cite{2018arXiv180506938A}, a este efecto se le conoce 
como galaxy assembly bias 
\cite{2006ApJ...639L...5Z}\cite{2007MNRAS.374.1303C}\cite{2016MNRAS.460.3100C}, 
la evidencia observacional y teórica para este efecto es aún controvertido y 
esta en debate. Desde el punto de vista observacional algunos estudios afirman 
la detección de este efecto 
\cite{1538-4357-638-2-L55}\cite{2014MNRAS.443.3107L}\cite{2017MNRAS.472.2504T}, 
pero  otros trabajos muestran que este efecto es pequeño e incluso 
insignificante sobre las propiedades galácticas 
\cite{2007ApJ...664..791B}\cite{2017MNRAS.468.3251D}. Una forma de estudiar las 
predicciones teóricas para el galaxy bias es por medio del análisis de 
diferentes modelos de formación de galaxias en simulaciones cosmológicas 
hidrodinámicas \cite{2018arXiv180506938A}.\\


El mejor método para entender cómo las galaxias se forman y evolucionan en el 
contexto cosmológico son las simulaciones numéricas hidrodinámicas de 
N-cuerpos. Estas simulaciones tienen la capacidad de mostrar en detalle la 
estructura interna de halos de materia oscura  y se han convertido en la 
herramienta fundamental para la prueba de modelos de formación de estructura y 
hacer predicciones teóricas precisas del acumulamiento de galaxias, formación y 
evolución de galaxias en sus halos y propiedades galácticas. Gracias al 
incremento en el poder computacional y a la creación de nuevos algoritmos, se 
han creado detalladas simulaciones hidrodinámicas de N-cuerpos de gran 
resolución. Una de ellas y la de mayor resolución es 
Illustris\footnote{http://www.illustris-project.org/} e Illustris 
TNG\footnote{http://www.tng-project.org/}. El proyecto Illustris es un conjunto 
de simulaciones hidrodinámicas cosmológicas que incluyen formación de galaxias 
\cite{2014MNRAS.444.1518V} dentro de un volumen de $75h^{-1}MPC$ y desarrollado 
con AREPO \cite{2010MNRAS.401..791S} en el contexto del $\Lambda CDM$, la 
fuerza gravitacional es calculada por el método Tree-PM. Las simulaciones del 
proyecto Illustris tienen tres diferentes resoluciones (Illustris-1, 
Illustris-2, Illustris-3) Illustris-1 es la de mayor resolución debido a que 
cubre mayores rangos de masa y sigue la evolución de partículas de materia 
oscura, partículas estelares y gas desde $z=127$ a $z=0$. Esta simulación ha 
reproducido diferentes propiedades observacionales como la distribución de 
morfologías de galaxias del diagrama de Hubble, evolución de la tasa de 
formación estelar hasta $z=8$ \cite{2014MNRAS.445..175G}. Illustris es un 
proyecto de libre acceso a sus bases de datos. Es posible obtener información 
de catálogos de galaxias a diferentes $z$ \cite{2015A&C....13...12N}; la última 
versión de esta simulación es Illustris-TNG\footnote{Datos no públicos por 
ahora} en ella se incluyen efectos como vientos galácticos, 
magnetohidrodinámica y una nueva implementación de retroalimentación cinética 
de agujeros negros supermasivos. Se puede estudiar funciones de correlación no 
lineales y espectros de potencia de materia oscura, galaxias y halos a muy 
grande escala y con esto hacer predicciones del sesgo para distintas muestras 
de galaxias \cite{2018MNRAS.475..676S}, respecto a la formación de galaxias. 
Illustris-TNG introduce simulaciones a gran escala 
gravedad-magnetohidrodinámica \cite{2018MNRAS.473.4077P}.  




%\nocite{oreg,schn,pond,smith,marg,hunn,advi,koha,mouse}

%%%%%%%%%%%%%%%%%%%%%%%%%%%%%%%%%%%%%%%%%%%%%%
%%                                          %%
%% Backmatter begins here                   %%
%%                                          %%
%%%%%%%%%%%%%%%%%%%%%%%%%%%%%%%%%%%%%%%%%%%%%%

%\begin{backmatter}

%\section*{Competing interests}
%  The authors declare that they have no competing interests.

%\section*{Author's contributions}
%    Text for this section \ldots

%\section*{Acknowledgements}
%  Text for this section \ldots
%%%%%%%%%%%%%%%%%%%%%%%%%%%%%%%%%%%%%%%%%%%%%%%%%%%%%%%%%%%%%
%%                  The Bibliography                       %%
%%                                                         %%
%%  Bmc_mathpys.bst  will be used to                       %%
%%  create a .BBL file for submission.                     %%
%%  After submission of the .TEX file,                     %%
%%  you will be prompted to submit your .BBL file.         %%
%%                                                         %%
%%                                                         %%
%%  Note that the displayed Bibliography will not          %%
%%  necessarily be rendered by Latex exactly as specified  %%
%%  in the online Instructions for Authors.                %%
%%                                                         %%
%%%%%%%%%%%%%%%%%%%%%%%%%%%%%%%%%%%%%%%%%%%%%%%%%%%%%%%%%%%%%

% if your bibliography is in bibtex format, use those commands:
\bibliographystyle{bmc-mathphys} % Style BST file (bmc-mathphys, vancouver, spbasic).
\bibliography{bmc_article}      % Bibliography file (usually '*.bib' )
% for author-year bibliography (bmc-mathphys or spbasic)
% a) write to bib file (bmc-mathphys only)
% @settings{label, options="nameyear"}
% b) uncomment next line
%\nocite{label}

% or include bibliography directly:
% \begin{thebibliography}
% \bibitem{b1}
% \end{thebibliography}

%%%%%%%%%%%%%%%%%%%%%%%%%%%%%%%%%%%
%%                               %%
%% Figures                       %%
%%                               %%
%% NB: this is for captions and  %%
%% Titles. All graphics must be  %%
%% submitted separately and NOT  %%
%% included in the Tex document  %%
%%                               %%
%%%%%%%%%%%%%%%%%%%%%%%%%%%%%%%%%%%

%%
%% Do not use \listoffigures as most will included as separate files


\end{document}
